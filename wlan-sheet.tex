\documentclass{article}
\usepackage[utf8]{inputenc}
\usepackage{fourier}
\usepackage{float}
\usepackage[hidelinks]{hyperref}
\usepackage{comment}
\usepackage{qrcode}
\pagenumbering{gobble}

\usepackage[swedish, english]{babel}
\babeltags{sv = swedish, en = english}

\renewcommand{\textsv}[1]{}
%\renewcommand{\texten}[1]{}

\newcommand{\ssid}{foo}
\newcommand{\password}{bar}



\title{\Huge{
\texten{Login details for WiFi}
\textsv{Inloggningsuppgifter för WiFi}
}}

\date{}
\author{}

\begin{document}
\maketitle

\begin{center}
\qrcode[height=3in]{WIFI:S:\ssid;T:WPA;P:\password;;}
\\
\vspace{.2in}
\begin{itemize}

\item\textbf{iOS:}
\texten{From version 11 of iOS (planned to be released in the fall of 2017) you can scan login details for WiFi with the built-in camera app.}
\textsv{Från version 11 av iOS (planeras släppas hösten 2017) kan man skanna inloggningsuppgifter för WiFi med den inbyggda kamera-appen.}

\item\textbf{Android:}
\texten{Use the app}
\textsv{Använd appen}
\emph{\href{https://play.google.com/store/apps/details?id=com.google.zxing.client.android}{Barcode Scanner}}
\texten{from}
\textsv{från}
\emph{ZXing Team}
\texten{available in}
\textsv{som finns i}
\emph{Play Store}
\texten{and choose}
\textsv{och välj}
\emph{Connect to Network}
\texten{after scanning the code.}
\textsv{efter att koden skannats för att ansluta.}

\item\textbf{\texten{Others}\textsv{Övriga}:}
\texten{Use the login details below.}
\textsv{Använd inloggningsuppgifterna nedan.}
\end{itemize}
\end{center}

\begin{table}[H]
\centering
\huge
\begin{tabular}{ll}
\textbf{SSID:}                               & \ssid  \\
\textbf{\texten{Password}\textsv{Lösenord}:} & \password
\end{tabular}
\end{table}

\end{document}

