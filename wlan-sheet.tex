\documentclass{article}
\usepackage[utf8]{inputenc}
\usepackage{fourier}
\usepackage{fontawesome}
\usepackage{float}
\usepackage[hidelinks]{hyperref}
\usepackage{qrcode}
\pagenumbering{gobble}

\usepackage[swedish, english]{babel}
\babeltags{sv = swedish, en = english}

\renewcommand{\textsv}[1]{}
%\renewcommand{\texten}[1]{}

\newcommand{\type}{WPA}
\newcommand{\ssid}{foo}
\newcommand{\password}{bar}
\newcommand{\authtype}{WPA}

\newcommand{\webadminurl}{http://192.168.1.1}
\newcommand{\webadminuser}{admin}
\newcommand{\webadminpass}{1234}

\selectlanguage{english}


\title{\Huge{\faWifi\,
\texten{WiFi Configuration}
\textsv{WiFi-konfiguration}
}}

\date{}
\author{}

\begin{document}
\maketitle

\begin{center}
\qrcode[height=3in]{WIFI:S:\ssid;T:\type;P:\password;;}
\\
\vspace{.1in}

\begin{large}
\begin{itemize}

\item\faApple\,\textbf{iOS:}
\texten{From version 11 of iOS (planned to be released in the fall of 2017) you can scan QR WiFi configuration with the built-in camera app.}
\textsv{Från version 11 av iOS (planeras släppas hösten 2017) kan man skanna QR-koder som innehåller WiFi-konfiguration med den inbyggda kamera-appen.}
\texten{To connect, tap on the \emph{Join ``\ssid '' Network} notification.}

\item\faAndroid\,\textbf{Android:}
\texten{Use the app}
\textsv{Använd appen}
\emph{\href{https://play.google.com/store/apps/details?id=com.google.zxing.client.android}{Barcode Scanner}}
\texten{from}
\textsv{från}
\emph{ZXing Team}
\texten{available in}
\textsv{som finns i}
\emph{Play Store}
\texten{and tap}
\textsv{och välj}
\emph{Connect to Network}
\texten{after scanning the QR WiFi configuration.}
\textsv{efter att QR-koden med WiFi-konfigurationen skannats för att ansluta.}

\item\faDesktop\,\textbf{\texten{Other}\textsv{Övriga}:}
\texten{Use the WiFi configuration below.}
\textsv{Använd WiFi-konfigurationen nedan.}

\end{itemize}
\end{large}
\end{center}

\begin{table}[H]
\centering
\huge
\begin{tabular}{ll}
\textbf{SSID:}                               & \ssid     \\
\textbf{\texten{Password}\textsv{Lösenord}:} & \password \\
\textbf{\texten{Security}\textsv{Säkerhet}:} & \type
\end{tabular}
\end{table}

\begin{center}
\huge\trwebadmin\\
\url{\webadminurl}\\
\vspace{.2in}
\qrcode[height=3in]{\webadminurl}
\end{center}

\begin{table}[H]
\centering
\huge
\begin{tabular}{ll}
    \textbf{\trusername:} & \webadminuser  \\
    \textbf{\trpassword:} & \webadminpass
\end{tabular}
\end{table}



\end{document}

